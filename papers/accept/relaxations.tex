\section{Analysis and Relaxations}
\label{accept:sec:relaxations}

\sysname takes an annotated program and applies a set of program transformations
to code that affects only data marked approximate.  We call these
transformations \term{relaxations} because they trade correctness for
performance.
%
To determine relaxation opportunities from type annotations, \sysname uses
an analysis called \emph{\precisepurity}.
%
This section describes \sysname's implementations of several program relaxations
drawn from the literature and how
\precisepurity analysis makes them safe.
%
As a framework for approximation, \sysname is extensible to
relaxations beyond those we describe here.

\iffalse % too wordy, no new information in this para (dan)
In contrast to traditional compiler optimizations,
\emph{program relaxations} are allowed to change the behavior of the input
program to improve performance. A broad array of previous research has
explored many different kinds of program relaxations~\cite{perforation,
quickstep, dubstep, races-ibm, green}. In this work, we seek to bring
programmer-directed \emph{safety guarantees} to these relaxation techniques.
Specifically, we implement a compiler analysis that determines whether
a code region has precise effects.
Individual program-relaxation strategies use this analysis to allow only those
transformations that have exclusively approximate externally visible effects.
These constraints uphold \sysname's noninterference guarantee, which states that
it must avoid perturbing precise data while optimizing the approximate part of
the program.
\fi

\subsection{\Precisepurity Analysis}
\label{accept:sec:precise-purity}

\sysname provides a core program analysis that client optimizations use to
ensure safety.
This analysis must reconcile a fundamental difference between the language's
safety guarantees and the transformation mechanisms:
the programmer specifies safety in terms of fine-grained annotations on
individual data elements, but program relaxations affect coarse-grained regions
of code such as loop bodies or entire functions.
Rather than resort to opaque and error-prone code-centric annotation,
\sysname bridges this gap by analyzing the side effects of coarse-grained code
regions.

\sysname's analysis library determines whether it is safe to approximate a
region of code.
Specifically, the \precisepurity analysis checks,
for a region of interest (e.g., a loop body), whether its side
effects are exclusively approximate or may include precise data---in other
words, whether it is \emph{pure with respect to precise data}.
%
\Precisepurity is the key criterion for whether a relaxation can
apply.  In
\sysname, every relaxation strategy consults the \precisepurity
analysis and optimizes only
\precisepure code.
%
A region is \precisepure if it:
\begin{itemize}
\item contains no stores to precise
variables that may be read outside the region;
\item does not call any functions that are not \precisepure; and
\item does not include an unbalanced synchronization statement (locking without
unlocking or vice versa).
\end{itemize}
The analysis begins with the conservative
assumption that the region is not \precisepure and asserts otherwise only if it
can prove \precisepurity.
Functions whose definitions are not available are conservatively considered
not \precisepure.
This includes standard-library functions, such as \lil{printf}, where input
and output make code unsafe to approximate.

For example, this code:
%
\begin{lstlisting}
int p = ...;
APPROX int a = p * 2;
\end{lstlisting}
%
is \precisepure if and only if the variable \lil{p} is never read outside this
code region. External code may, however, read the variable \lil{a} since it is
marked as approximate.
%
Together with the information-flow type system, the \precisepurity restriction
ensures that code transformations influence only approximate data.
Since only the approximate value \lil{a} escapes the \precisepure block above,
dependent code must also be marked as \lil{APPROX} to obey the typing rules:
any code that treats \lil{a} as precise is a type error.
Optimizations that affect only \precisepure code uphold \sysname's
contract with the programmer: that approximation must affect only variables
explicitly marked as approximate.

We implement the core \precisepurity analysis
conservatively using SSA definition--use chains and a simple pointer-escape
analysis.  Section~\ref{accept:sec:impl} gives more implementation details.

\iffalse  % Seems out of place to me now. -ALDS
Each program-relaxation strategy uses the shared \precisepurity analysis to
identify \emph{relaxation-opportunity sites:} places where the transformation
can apply. When \sysname finds an opportunity site, it registers the site along
with any relevant parameters in its catalog of such sites. The autotuner then
uses this catalog to enable a subset of relaxation opportunities and set their
parameters.  Section~\ref{accept:sec:autotuner} describes the autotuning process in more detail.
\fi

\subsection{Target Region Selection}
\label{accept:subsec:regions}

\begin{algorithm}[tb]
  \DontPrintSemicolon
  \SetArgSty{textrm}
  \KwIn{function $f$}
  \KwOut{set of \precisepure regions $R$ in $f$}

  \ForEach{basic block $B$ in $f$} {
    \ForEach{block $B'$ strictly post-dominated by $B$} {
      \If{$B'$ dominates $B$}{
        $region \gets \text{formRegionBetween}(B', B)$ \;
        \label{accept:line:regionSel:dfs}
        \If{$region$ is \precisepure} {
          $R \gets R \cup \{region\}$\;
        }
      }
    }
  }
\caption{Candidate region selection.}
\label{accept:alg:regionSel}
\end{algorithm}

Accelerator-style program transformations work best when they target larger
regions of code.
To help optimizations identify profitable targets,
\sysname can enumerate a
function's replaceable approximate code regions.
%
A \term{candidate region} is a set of instructions that is \precisepure, forms
control flow with a single entry and a single exit, and has identifiable live-ins
and live-outs.
%
Client optimizations, such as the neural acceleration described in
\secref{sec:npu}, can enumerate the candidate regions in a program to attempt
optimization.  \Precisepurity analysis enables region selection by proving
that chunks of code are cleanly separable from the rest of the program.

Region selection meets the needs of accelerators that do not access memory
directly and therefore require statically identifiable inputs and outputs;
patterns such as dynamic array updates cannot be offloaded.  The same analysis
can be adapted to superoptimizers and synthesizers that need to operate on
delimited subcomputations.
%
For example, an accuracy-aware superoptimizer such as
STOKE~\cite{stoke-fp} could use \sysname's region selection to search for
tractable optimization targets in a large program.
Each fragment could be optimized independently
and spliced back into the program.

Algorithm~\ref{accept:alg:regionSel} shows how \sysname enumerates candidate regions.
The algorithm uses dominance and post-dominance sets to
identify pairs of basic blocks $B_1$ and $B_2$
where $B_1$ dominates $B_2$ and $B_2$ post-dominates $B_1$.
The portion of the control-flow graph between these pairs represent all the
single-entry, single-exit portions of a function.
For a function with $n$ blocks, the enumeration needs $n^2$ \precisepurity
checks in the worst case---but typically fewer because the LLVM compiler
infrastructure pre-computes the dominator and post-dominator trees.


\subsection{Safe Approximate Relaxations}

To demonstrate \sysname's flexibility as a framework, we implement three
approximation strategies from the literature using \precisepurity analysis.

\subsubsection{Loop Perforation}
Sidiroglou \etal propose \emph{loop perforation},
which exploits the fact that many programs tolerate some skipping of loop
iterations without significant quality degradation~\cite{perforation}.
A perforated loop includes a parameter, the \term{perforation factor}, that
governs how often an iteration can be skipped at run time.

\sysname considers a loop safe to perforate if its body is
\precisepure and free of early exits (i.e., \lil{break}
statements), which can cause nontermination if skipped.
To perforate a loop, \sysname inserts a counter and code to increment and
check it in each loop iteration.
To minimize the overhead of loop perforation, \sysname
requires the perforation factor $p$ to be a power of two to enable bitwise tests
against the counter.  The loop body executes once every $p$ iterations.

\subsubsection{Synchronization Elision}
In parallel programs, inter-thread synchronization constructs---locks,
barriers, semaphores, etc.---are necessary for program predictability but
threaten scalability.
Recent research has proposed
to strategically reduce
synchronization in approximate programs~\cite{rinard-hotpar, quickstep, dubstep, races-ibm}.
Even though removing synchronization can add data races and other
nondeterminism to previously race-free or deterministic programs, this recent
work has observed that the ``incorrectness'' is often benign:
the resulting lost updates and atomicity violations can
sometimes only slightly change the program's output.

\sysname can elide calls to locks (mutexes) and
barriers from the pthreads library.
To permit the elision of a lock
acquire--release pair, \sysname requires that the critical section---the code
between the acquire and release---be \precisepure.
To elide
\lil{pthread_barrier_wait()} synchronization, \sysname looks for pairs of calls
whose intervening code is \precisepure, in such cases removing the \emph{first}
call (the second call remains to delimit the end of the region).

\subsubsection{Neural Acceleration}
\label{accept:sec:npu}

Recent work has shown how to accelerate approximate programs with hardware
neural networks~\cite{benchnn, temam-isca, temam-isca13}.
\textit{Neural acceleration} uses profiled inputs and outputs from a region of
code to train a neural network that mimics the code.
The original code is then replaced with an invocation of an
efficient hardware accelerator implementation, the Neural Processing Unit
(NPU)~\cite{npu, anpu, snnap}.
But the technique has thus far required manual identification of candidate
code regions and insertion of offloading instructions.
\sysname automates the process.

\sysname implements an automatic neural acceleration transform
that uses an existing configurable neural-network implementation
for an on-chip field-programmable gate array (FPGA)~\cite{snnap}.
\sysname uses approximate region selection (\Sref{subsec:regions}) to
identify acceleration targets, then trains a neural network on
execution logs for each region.
It then generates code to offload executions of the identified region to the
accelerator.
The offload code hides invocation latency by constructing batched invocations
that exploit the high-bandwidth interface between the CPU and FPGA.
We target a commercially available FPGA-augmented system on a chip (SoC) and
do not require specialized neural hardware.

\subsubsection{Other Client Relaxations}
The three optimizations above demonstrate \sysname's breadth as a
framework for realizing ideas from approximate-computing research.
We have also used \sysname to prototype
two other optimizations, not described here:
an approximate alias analysis that unlocks secondary compiler optimizations such as
loop-invariant code motion and vectorization for approximate data, and
approximate strength reduction that aggressively replaces expensive arithmetic
operations with cheaper shifts and masks that are not exactly equivalent.
Other optimizations from the literature are also amenable to \sysname's
architecture, including approximate parallelization~\cite{quickstep},
float-to-fixed conversion~\cite{torftf},
bit-width reduction~\cite{bitwidthred, precimonious}, GPU pattern replacement~\cite{paraprox},
and alternate-algorithm selection~\cite{green, petabricks}.

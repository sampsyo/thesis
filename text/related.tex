Approximate computing research combines insights from hardware engineering,
architecture, system design, programming languages, and even application
domains like machine learning.
This chapter summarizes research on implementing, exploiting,
controlling, and reasoning about approximation in computer systems.
To confine the scope, the survey focuses on work that exposes error to
applications (unlike fault tolerance, which seeks to \emph{hide}
errors),
and work that is in some sense general (not, for example, a new approximation
strategy for one specific graphics algorithm).

\section{Application Tolerance Studies}
\label{sec:related:studies}

Many authors have identified the property of error tolerance in existing
``soft''
applications. A large class of studies have examined this property by
injecting errors into certain parts of applications and assessing the
execution quality in terms of both crashes and output fidelity~\cite{li06,
li07, li08, dekruijf-selse09, wong-selse06, palem-arcs, freton, besteffort,
yeh, thaker-iiswc06, efc, llfi, chippa-dac}.
Related
studies have evaluated error-resilience in
integrated circuit designs~\cite{breuer, scalable-effort-hardware}.
This category of study repeatedly finds that
different parts of the
application have different impacts on reliability and fidelity.
Some conclude
that there is a useful distinction between critical and non-critical program
points, typically instructions~\cite{palem-arcs, thaker-iiswc06, flikker,
llfi}.
This conclusion reflects
the safety principle in \secref{princ:safety}:
certain program components, especially those involved in control flow, need to
be protected from all influence from approximation.

Much of the work on application robustness tends to assume an existing,
domain-specific notion of ``quality'' for each
application.
As the principle in \secref{princ:appspecific} suggests, these quality metrics
need careful consideration: one quality metric is not necessarily just as good
as another.
Recent work has proposed guidelines for measuring quality
rigorously~\cite{wddd-quality}.


\section{Exploiting Resilience in Architecture}

One category of techniques for approximation examines efficiency--accuracy
trade-offs in hardware.
These techniques can lead to gains in energy, performance, manufacturing
yield, or verification complexity.

\subsection{Functional Units}

Researchers have designed floating-point units
that dynamically adapt mantissa width~\cite{bitwidthred, hierarchfpu}, ``fuzzily'' memoize
similar arithmetic computations~\cite{fuzzymemo}, or tolerate timing
errors~\cite{kumarhpca, hizli, metafunctions}.
Alternative number
representations work in tandem with relaxed functional units to bound the
numerical error that can result from bit flips~\cite{stanleymarbell}.

The VLSI community has paid particular attention to variable-accuracy adder
designs, which are allowed to yield incorrect results for some minority of
input combinations~\cite{uva-adder, palem-adders, impact, adder-metrics,
configurable-adder, adder-iccad13, adder-tcad, adder-optimal, adder-dac12,
adder-isic09, adder-date08}.

\subsection{Memory}

SRAM
structures spend significant static power on retaining data, so they represent
another opportunity for fidelity trade-offs~\cite{hybrid-sram, sramerrors,
partially-forgetful}. Similarly,
DRAM structures can reduce the power spent on refresh cycles where bit flips
are allowed~\cite{flikker, sparkk}.
In persistent memories where storage cells can wear out, approximate systems
can reduce the number of bits they flip to lengthen the useful device
lifetime~\cite{fang-pcm}.
Similarly, low-power writes to memories like flash can exploit its
probabilistic properties while hiding them from software~\cite{halfwits,
powerfade, flash-retention-relax}.
Spintronic memories exhibit similarly favorable trade-offs between access cost
and error~\cite{spintronic-approx}.

These memory approximation techniques typically
work by exposing
soft errors and other analog effects.
Recent work in security and privacy has exploited these variability-related
errors to fingerprint hardware and deanonymize users~\cite{deanondram}.

\subsection{Circuit Design}

A broad category of work has proposed general techniques for making quality trade-offs
when synthesizing and optimizing general hardware
circuits~\cite{lossysynthesis, palem-pruning, rahimi, axilog, miao-thesis, synthesis-date14, venkataramani-date13, venkataramani-dac12}.
Other tools focus on analyzing approximate circuit
designs~\cite{venkatesan-iccad11, tziantzioulis-dac15}.

Near-threshold voltage domains also present a new opportunity for embracing
unpredictable circuit operation~\cite{soft-ntc}.

\subsection{Relaxed Fault Tolerance}

As a dual to adding errors in some circuits, some researchers have
explored differential fault protection in the face of universally unreliable
circuits. As process sizes continue to shrink, it is likely that reliable
transistors will become the minority; redundancy and checking will be
necessary to provide reliable operation~\cite{li-asplos08}. Circuit design
techniques have been proposed that reduce the cost of redundancy by providing
it selectively for certain instructions in a CPU~\cite{wreft} or certain
blocks in a DSP~\cite{unequal-protection, ant, micropower-dsp}.
Researchers at Wisconsin have applied relaxed error tolerance at the
architecture level to GPUs~\cite{palframan-gpu}.
Other work has used criticality information to selectively allocate
software-level error detection and correction
resources~\cite{khudia-tolerance, shi-cal}.

\subsection{Microarchitecture}

Aside from designing fundamentally approximate circuits, a different direction
introduces approximation by relaxing traditional microarchitectural
mechanisms.
Notably, ``soft coherence'' relaxes intercore
communication~\cite{softcoherence},
and load value approximation~\cite{lva-sanmiguel, lva-thwaites} approximates
numerical values instead of fetching them from main memory on cache misses.

Recent work has proposed system organizations that apply approximation at a
coarser grain.
One set of techniques that use external monitoring to
allow errors even in processor control logic~\cite{martonosi-date, commguard}.
Other approaches compose separate processing units with different levels of
reliability~\cite{ersa}.
Duwe~\cite{duwe-thesis} proposes run-time coalescing of approximate and
precise computations to reduce the overhead of switching between modes.
Other work allocates approximation among the lanes of a SIMD
unit~\cite{tabsh}.

\subsection{Stochastic Computing}

\emph{Stochastic computing} is an alternative computational model where values
are represented using probabilities~\cite{pcmos, pcmos-cacm,
palem-dac-position, stochasticproc, storm, lyric, mansinghka-circuits}.
For example, a wire could carry a random sequence of bits, where the wire's
value corresponds
to the probability that a given bit is a 1.
Multiplication can be implemented in this model using a single \emph{and}
gate.


\section{Exploiting Resilience with Program Transformations}
\label{sec:related:software}

Aside from hardware-level accuracy trade-offs, there are opportunities for
adapting \emph{algorithms} to execute with varying precision. Algorithmic
quality--complexity trade-offs are not new, but recent work has
proposed
tools for \emph{automatically} transforming programs to exhibit these
trade-offs.
Transformations include removing portions of a program's dynamic execution
(termed \emph{code perforation})~\cite{perforation}, unsound
parallelization of serial programs~\cite{quickstep}, eliminating
synchronization in parallel programs~\cite{dubstep, races-ibm, hogwild,
forgiving-parallel},
identifying and adjusting parameters that control output
quality~\cite{dynamicknobs}, randomizing portions of deterministic
programs~\cite{zhu-popl12, sasa-sas11}, dynamically choosing between
different programmer-provided implementations of the same
specification~\cite{green, virus, petabricks, taco-soc, ansel-autotuning,
scalable-classifier}, and replacing subcomputations with invocations
of a trained neural network~\cite{npu}.

Some work on algorithmic approximation targets specific hardware: notably,
general-purpose GPUs~\cite{paraprox, sage, herding, neuralgpu}.
In a GPU setting, approximation strategies benefit most by optimizing for
memory bandwidth and control divergence.

Recently, a research direction has developed in \emph{automated program
repair} and other approaches to heuristically patching software according to
programmer-specified criteria.
These techniques are typically approximate in that they abandon a traditional
compiler's goal of perfectly preserving the original program's semantics.
Notably, Schulte \etal~\cite{schulte} propose to use program evolution to
optimize for energy.

Precimonious~\cite{precimonious} addresses the problem of choosing appropriate
floating-point widths, which amount to a trade-off between numerical accuracy
and space or operation cost.
Similarly, STOKE's floating-point extension~\cite{stoke-fp} synthesizes new
versions of floating-point functions from scratch to meet different accuracy
requirements with optimal efficiency.

Neural acceleration is a recent technique treading code as a black box and
transforming it to a neural-network representation~\cite{npu, emeuro, benchnn,
temam-isca}.
It is, at its core, an algorithmic transformation, but it integrates tightly
with hardware support: a digital accelerator~\cite{npu}, analog
circuits~\cite{anpu}, FPGAs~\cite{snnap},
GPUs~\cite{neuralgpu}, or, recently, new analog substrates using
resistive memory~\cite{rram-npu} or memristors~\cite{memristor-npu}.
See \secref{npu} for a more detailed overview of neural acceleration.


\section{Exploiting Resilience in Other Systems}

While architecture optimizations and program transformations dominate the
field of proposed exploitations of approximate software, some recent work has
explored the same trade-off in other components of computer systems.

Network
communication, with its reliance on imperfect underlying channels, exhibits
opportunities for fidelity trade-offs~\cite{softcast, luo-globecom, apex,
smpmup2006}. Notably, SoftCast~\cite{softcast} transmits images and video by
making the signal magnitude directly proportional to pixel luminance.

BlinkDB,
a recent instance of research on \emph{approximate query answering},
is a database system that can respond to queries that include a required
accuracy band on their output~\cite{blinkdb}.

Uncertain{\textless}T{\textgreater}~\cite{uncertaint} and Lax~\cite{lax}
propose to expose the approximate, probabilistic behavior of sensors at the
language and API levels.

In a distributed system or a supercomputer, approximation techniques can
eschew redundancy and recovery for efficiency~\cite{dekruijf-icpp}.


\section{Languages for Expressing Approximation}

Recently, language constructs that express and constrain
approximation have become a focus in the programming-languages research
community.
Relax~\cite{relax} is a language with ISA support for tolerating architectural
faults in software.
Rely~\cite{rely} uses specifications that relate the reliability of the input
to an approximate region of code to its outputs.

A related set of recent approximate-programming tools attempt to \emph{adapt}
a program to meet accuracy demands while using as few resources as possible.
Chisel~\cite{chisel} is an extension to Rely that searches for the subset of
operations in a program that can safely be made approximate.
ExpAX~\cite{expax-tr} finds safe-to-approximate operations automatically and
uses a metaheuristic to find which subset of them to actually approximate.

Some other programming systems that focus on energy efficiency include
approximation ideas:
Eon~\cite{eon} is a language for long-running embedded systems that can drop
tasks when energy resources are low,
and the Energy Types language~\cite{energytypes} incorporates a variety of
strategies for expressing energy requirements.


\section{Programmer Tools}

Aside from programming languages, separate programmer tools can help analyze
and control the effects of approximation.

A quality-of-service profiler helps programmers identify parts of programs
that may be good targets for approximation techniques~\cite{qosprof}.
Conversely, debugging tools can identify components where approximation is too
aggressive~\cite{approxdebug}.

Some verification tools and proof systems help
the programmer prove relationships between the original program and a
candidate relaxed version~\cite{carbin-pldi, carbin-races, carbin-pepm,
rice-transformation-semantics}.

As an alternative to statically bounding errors, dynamic techniques can
monitor quality degradation at run time.
The critical challenge for these techniques is balancing detection accuracy
with the added cost, which takes away from the efficiency advantages of
approximation.
Some work has suggested that programmers can provide domain-specific checks on
output quality~\cite{lwc, approxdebug}.
Recent work has explored automatic generation of error detectors~\cite{rumba}.
A variety of techniques propose mechanisms for run-time or profiling feedback to adapt
approximation parameters~\cite{dynamicknobs, green, approxit, ansel-autotuning}.


\section{Probabilistic Languages}

One specific research direction, \emph{probabilistic programming languages},
focuses on expressing statistical models, especially for machine
learning~\cite{BBGR13, wingate-lightweight, church, chaganty, pfeffersample,
probdsl, koller, sriram-pldi}.
The goal is to enable efficient statistical inference over arbitrary models
written in the probabilistic programming language.

Earlier work examines the semantics of probabilistic behavior in more
traditional programming models~\cite{kozen}.
Similarly, the probability monad captures a variable's discrete probability
distribution in functional programs~\cite{pmonad}.
Statistical model checking tools can analyze programs to prove statistical
properties~\cite{legay10, KNP11}.
Recently, \citet{uncertaint} proposed a construct for explicitly representing
probability distributions in a mainstream programming language.


\section{Robustness Analysis}

As the studies in Section~\ref{sec:related:studies} repeatedly find, error
tolerance varies greatly in existing software, both within and between
programs.
Independent of approximate computing, programming-languages researchers have
sought to identify and enhance error resilience properties.

SJava analyzes programs to prove that
errors only temporarily disrupt the execution path of a program~\cite{sjava}.
Program smoothing~\cite{smoothing-cav, smoothing-pldi, smoothing-fse} and
``robustification''~\cite{robustification} both find continuous, mathematical
functions that resemble the input--output behavior of numerical programs.
Auto-tuning approaches can help empirically identify error-resilient
components~\cite{asac}.
Finally, Cong and Gururaj describe a technique for automatically
distinguishing between critical and non-critical instructions for the purpose
of selective fault tolerance~\cite{cong-iccad}.
